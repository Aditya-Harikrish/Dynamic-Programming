\documentclass[12pt, a4paper]{article}
\usepackage[UKenglish]{babel}
\usepackage[utf8]{inputenc}
\usepackage{mathtools} % an extension of amsmath
\usepackage{amsthm, amssymb, amsfonts, amstext}
\usepackage{longtable}
\usepackage{tocloft}
\usepackage{geometry}
\usepackage{algpseudocode}
\usepackage{multicol}
\geometry{margin=27mm}
\usepackage{listings}
\usepackage{xcolor,colortbl}
\usepackage{wrapfig}
\usepackage{graphicx}
\usepackage{array}
\usepackage{tabularx}
\usepackage{makecell}
\renewcommand\theadalign{bc} % to leave lines inside a table
\renewcommand\theadgape{\Gape[4pt]}
\renewcommand\cellgape{\Gape[4pt]}
\definecolor{codegreen}{rgb}{0,0.6,0}
\definecolor{codegray}{rgb}{0.5,0.5,0.5}
\definecolor{codepurple}{rgb}{0.58,0,0.82}
\definecolor{backcolour}{rgb}{0,0,0}
\definecolor{pageBG}{rgb}{0,0,0}
\usepackage{hyperref}
\hypersetup{
    colorlinks=true,
    linkcolor=blue,
    filecolor=magenta,
    urlcolor=cyan,
}
\lstdefinestyle{mystyle}{
    backgroundcolor=\color{backcolour},
    commentstyle=\color{yellow},
    keywordstyle=\color{red},
    numberstyle=\color{red},
    stringstyle=\color{codepurple},
    basicstyle=\ttfamily\footnotesize\color{codegreen},
    breakatwhitespace=false,
    breaklines=true,
    captionpos=b,
    keepspaces=true,
    numbers=left,
    numbersep=5pt,
    showspaces=false,
    showstringspaces=false,
    showtabs=false,
    tabsize=4
}
\lstset{style=mystyle}
\urlstyle{same}

\newtheorem{theorem}{Theorem}[section]
\theoremstyle{definition}
\newtheorem{definition}{Definition}[section]
\newtheorem*{claim}{Claim}
\theoremstyle{remark}
\newtheorem*{remark}{Remark}
% \numberwithin{equation}{section}
\newcommand{\textsb}{\textsubscript}

\title{Algorithms Analysis \& Design\\Project Proposal}
\author{Aditya Harikrish\\2020111009}
\date{}

\begin{document}
\maketitle
% \tableofcontents

\section{Introduction}
Dynamic programming is a widely used problem-solving tool that often greatly helps in reducing the effort needed to solve computational problems. It can also be unintuitive to beginners.

\section{Proposal}
I plan on making a beginner's guide to dynamic programming. To build up on the prerequisites, I will introduce time and space complexities, basic examples of dynamic programming where it outperforms some of the competing algorithms, such as the Fibonacci sequence. I will then formally define dynamic programming, explain how a solution could be intuitively and originally thought of and what kinds of problems can be solved using dynamic programming. Some of the problems / algorithms I wish to cover are:
\begin{enumerate}
    \item Fibonacci sequence
    \item Frog Jump
    \item Longest increasing subsequence
    \item Longest common subsequence
    \item Shortest common supersequence
    \item Minimum coins
    \item Longest palindromic substring
    \item Knapsack
    \item Tower of Hanoi (one of my all-time favourite problems)
    \item Egg dropping problem
    \item Bellman Ford Algorithm
    \item Floyd Warshall Algorithm
\end{enumerate}
I shall go through memoisation as well in the process. If possible, I would like to cover Optimal binary search trees.

\section{Timeline}
The tentative timeline is as follows. I would like to finish most of the theoretical part including the prerequisites, along with 5 or so problems by the first week of November.
I plan to finish the project by including more problems and if needed, polishing the theoretical parts by the third or fourth week of November.

\end{document}